\documentclass{article}

\usepackage[a4paper, top=2cm, bottom=2cm, left=3cm, right=3cm,
    marginparwidth = 1.75cm]{geometry}

\usepackage[portuguese]{babel}
\usepackage[utf8]{inputenc}
\usepackage{amsfonts}
\usepackage{amsmath}
\usepackage{amssymb}
\usepackage{systeme}

\usepackage{indentfirst}

\usepackage{graphicx}
\usepackage{tikz}

\title{Correção P2 de Fundamentos da Matemática}
\author{Pedro Luís Anghievisck}
\date{Junho - 2024}

\begin{document}
    \maketitle

    \tableofcontents
    \newpage

    \section{Correção do exercício 1}
    Sejam $f:A\rightarrow B$ e $g:B\rightarrow C$ funções e 
    conjuntos $C_1, C_2 \subset A$ e $Y_1, Y_2 \subset B$. 
    Identifique cada afirmação como verdadeira ou falsa. 
    Justifique:
    \subsection*{Item \emph{a}}
    \begin{equation*}
        \text{[ F ] : } f(C_1 \cup C_2) = f(C_1) \cup f(C_2)
        \text{ e } f(C_1 \cap C_2) = f(C_1) \cap f(C_2)
    \end{equation*}

    A afirmação anterior é falsa, pois podemos observar que
    $f$ pode não ser injetora, o que significa que $\exists x
    \in C_1$ e $x \notin C_2 : f(x) = f(y)$, onde $y \in C_2$ e
    $y \notin C_1$ o que implica que, como $x$ e $y$ não estão
    em $C_1\cap C_2$, $f(C_1\cap C_2) \neq f(C_1) \cap f(C_2)$. 

    Visto que $x\in C_1 \rightarrow f(x)\in f(C_1)$ e $y\in C_2
    \rightarrow f(y) \in f(C_2)$, mas $x\notin C_2 \rightarrow
    f(x)\notin f(C_2)\rightarrow f(x)\notin f(C_1\cap C_2)$ de
    mesma forma: $y\notin C_1 \rightarrow f(y)\notin f(C_1)
    \rightarrow f(y)\notin f(C_1\cap C_2)$. Por fim,
    sabemos que como $f(x) = f(y), f(x)$ e $f(y)\in f(C_1) \cap
    f(C_2)$.
    
    O que nos mostra que realmente: 
    \begin{equation*}
        f(C_1\cap C_2)\neq f(C_1)\cap f(C_2)
    \end{equation*}

    \section{Exercício 2 alternativo}
    Seja $\mathcal{F}$ o conjunto de todas as funções de
    $\mathbb{R}$ em $[0, +\infty[$. Mostre que a relação
    $f\leq g$ definida por
    \begin{equation*}
        f(x)\leq g(x),\;\; \forall x\in\mathbb{R},
    \end{equation*}
    é uma ordem parcial em $\mathcal{F}$ e determine o mínimo
    de $\mathcal{F}$ segundo essa ordem.

    Logo, para $\leq$ ser uma relação de ordem parcial, é preciso
    mostrar que dado dois elementos do conjunto, a relação deve
    ser reflexiva, transitiva e antiassimétrica, respectivamente:
    \begin{equation*}
        \begin{aligned}
            \forall x\in\mathbb{R}, \\
            \forall a(x)\in\mathcal{F}, a(x)\leq a(x)
        \end{aligned}
    \end{equation*}
    \begin{equation*}
        \begin{aligned}
            \forall a(x),b(x),c(x)\in\mathcal{F} : a(x)\leq b(x),
            b(x)\leq c(x)\rightarrow a(x)\leq c(x) \\
            \forall a(x),b(x)\in\mathcal{F} : a(x)\leq b(x)
            \text{ e }b(x)\leq a(x)\rightarrow a(x) = b(x)
        \end{aligned}
    \end{equation*}

    Note que para obtermos um mínimo precisamos achar uma função
    com valor fixo ao menor valor do domínio, que no nosso caso
    é o próprio $0$, visto que o contradomínio de todas as
    funções é $[0, +\infty[$, portanto, seja $f(x)\in\mathcal{F}
    : f(x) = 0$, $\forall x\in\mathbb{R}$ assim temos:
    \begin{equation*}
        f(x)\leq g(x),\;\;\forall g(x)\in\mathcal{F}\text{ e }
        \forall x\in\mathbb{R}\rightarrow f(x)<g(x)\text{ ou }
        f(x)=g(x)
    \end{equation*}

    \section{Correção do exercício 3 e item alternativo}
    Escolha UMA das umas afirmações abaixo e prove por indução.
    
    I) Para todo natural $n$, $9$ divide $4^n + 15n - 1$.
    
    II) Para todo natural $n$, $3$ divide $n^3 - n$.
    
    III) Para todo natural $n\neq 0$, $3^n$ divide
    $4^{3^{n-1}} - 1$
    \subsection*{Correção do item \emph{III}}
    Proposição:
    \begin{equation*}
        P(x): \forall x\in\mathbb{N}\setminus\{0\}
        3^x|4^{3^{x-1}}-1
    \end{equation*}

    Caso 0:
    \begin{equation*}
        3^1|4^{3^{1-1}}-1 = 3|4^1 - 1\rightarrow 3|3
    \end{equation*}

    Assumindo caso $P(n)$, façamos $P(n+1)$:
    \begin{equation*}
        \begin{aligned}
            3^{n+1}|4^{3^{n+1-1}}-1 = (3^n\cdot 3)|4^{3^n}-1 \\
            = 3^n\cdot3|(4^{3^{n-1}}\cdot4^{3}) - 1 =
            3^n|4^3\cdot(1^{1^{n-1}}\cdot1) - 1 \\
            = 3^n|4^{3^{n-1}} - 1 \rightarrow P(n)
        \end{aligned}
    \end{equation*}

    Assim, $P(n+1)$ vale, portanto, $\forall x\in\mathbb{N}
    \setminus\{0\}$, $3^x$ divide $4^{3^{x-1}}-1$

    \subsection*{Item alternativo \emph{II}}
    Proposição:
    \begin{equation*}
        P(x): \forall x\in\mathbb{N} \rightarrow 3|x^3 - x =
        3|x(x^2 - 1) = 3|x(x + 1)(x - 1)
    \end{equation*}

    Caso n = 0 vale pois:
    \begin{equation*}
        P(0):  3 | 0^3 - 0 = 3|0
    \end{equation*}

    Assumindo que $P(n)$, façamos $P(n + 1)$:
    \begin{equation*}
        \begin{aligned}
            P(n+1) = 3|(n + 1)[(n + 1) + 1][(n + 1) - 1] \\
            = 3|(n+1)(n+2)n = 3|(n^2 + 2n + n + 2)n \\
            = 3|n^3 + 3n^2 + 2n
        \end{aligned}
    \end{equation*}

    É possível observar que $3$ sempre dividirá $3n^2$, já que
    $n^2$ está sendo multiplicado por $3$, o que o torna um
    múltiplo de $3$, assim precisamos provar que $\forall
    n\in\mathbb{N}$, $n^3 + 2n$ é um múltiplo de $3$, pois
    todo múltiplo de $3$ mais outro múltiplo de $3$ resulta
    em um múltiplo de 3. Portanto precisamos provar que:
    \begin{equation*}
        B(x): 3|x^3 + 2x, \forall x\in\mathbb{N}
    \end{equation*}

    Assim, através do \emph{PIF} (Princípio da Indução Infinita), iremos
    provar $B(x)$.

    Caso n = 0:
    \begin{equation*}
        B(0): 3|0^3 + 2\cdot 0 = 3|0 
    \end{equation*}

    Assumindo $B(n)$, faremos $B(n+1)$:
    \begin{equation*}
        \begin{aligned}
            3|(n+1)^3 + 2\cdot (n+1) = 3|(n^3 + 3n^2 + 3n + 1) +
            (2n + 2) \\
            = 3|n^3 + 3n^2 + 5n + 3 = 3|n^3 + 5n \\
            = 3|n^3 + 2n + 3n = 3|n^3 + 2n \rightarrow
            \text{ $B(n)$ vale}
        \end{aligned}
    \end{equation*}

    Como visto acima, usamos mais uma vez do fato que múltiplos 
    de $3$ somados resultam em outro múltiplo de $3$, e chegamos
    a conclusão que queriamos. Assim provando $B(x)$ pelo
    \emph{PIF}, acabamos também provando $P(x)$. Logo, para todo
    $n\in\mathbb{N}$, $3$ divide $n^3 - n$.

    \section[Observação]{Exercício 4}
    Em minha prova, caiu a seguinte versão da questão:
    \begin{equation*}
        f((a,b),c) = f(a,b\cdot c)
    \end{equation*}
    
    E em minha resolução eu acabei provando e usando a outra
    versão:
    \begin{equation*}
        f(a,b)\cdot f(a,c) = f(a,b+c)
    \end{equation*}

    Por isso, não incluirei a resolução a outra versão da 4.
\end{document}