\documentclass{article}

\usepackage[a4paper, top=2cm, bottom=2cm, left=3cm, right=3cm,
    marginparwidth = 1.75cm]{geometry}

\usepackage[portuguese]{babel}
\usepackage[utf8]{inputenc}
\usepackage{amsfonts}
\usepackage{amsmath}
\usepackage{amssymb}
\usepackage{systeme}

\usepackage{indentfirst}

\usepackage{graphicx}
\usepackage{tikz}

\title{Anotações de Fundamentos da Matemática}
\author{Anghie}
\date{May 2024}

\begin{document}
    \maketitle

    \tableofcontents
    \newpage

    \section{Listas de Exercícios}
    \subsection{Lista 3}
    Lista sobre funções
    \subsection*{Exercício 1}
    Sejam $f:A\rightarrow B$ e $g:B\rightarrow C$ funções, $C_1$
    e $C_2$ subsconjuntos de $A$, e $Y_1$ e $Y_2$ subsconjuntos
    de $B$. Mostre as seguintes afirmações.
    \subsubsection*{1.a: $f(C_1\cup C_2) = f(C_1)\cup f(C_2)$}
    Resolução:
    \begin{equation*}
        \begin{aligned}
            \forall c_1\in C_1 \rightarrow f(c_1)\in f(C_1) \\
            \forall c_2\in C_2 \rightarrow f(c_2)\in f(C_2)
        \end{aligned}
    \end{equation*}

    Também temos que:
    \begin{equation*}
        \begin{aligned}
            \forall c_1\in C_1 \rightarrow c_1\in C_1\cup C_2 \\
            \forall c_2\in C_2 \rightarrow c_2\in C_1\cup C_2
        \end{aligned}
    \end{equation*}

    Portanto, como mostrado anteriormente:
    \begin{equation*}
        \begin{aligned}
            \forall x\in C_1\cup C_2 \rightarrow f(x)\in
            f(C_1\cup C_2) \\
            \forall x\in C_1\cup C_2 \rightarrow f(x)\in
            f(C_1)\cup f(C_2) \\
            \therefore \; f(C_1\cup C_2) = f(C_1)\cup f(C_2)
        \end{aligned}
    \end{equation*}

    \subsubsection*{1.b: $f(C_1\cap C_2) \subset f(C_1)\cap
    f(C_2)$\normalfont{. Mostre que a igualdade não vale sempre}}
    Resolução:
    \begin{equation*}
        \begin{aligned}
            \forall c_1\in C_1 \rightarrow f(c_1)\in f(C_1) \\
            \forall c_2\in C_2 \rightarrow f(c_2)\in f(C_2) \\
            x\in C_1\cap C_2 \rightarrow x\in C_1, C_2
        \end{aligned}
    \end{equation*}

    Temos:
    \begin{equation*}
        \begin{aligned}
            \forall x\in C_1\cap C_2 \rightarrow f(x)\in
            f(C_1\cap C_2) \\
            \forall x\in C_1\cap C_2 \rightarrow f(x)\in
            f(C_1)\cap f(C_2) \\
        \end{aligned}
    \end{equation*}

    \section{Cortes de Dedekind}
    \subsection{Soma de cortes}
    \subsubsection*{Elemento Neutro}
    Seja $C_r$ e $C_0$ em $\mathbb{Q}$, temos que $C_0$ é o 
    elemento neutro da soma, visto que, para todo
    $r\in\mathbb{Q}$, $C_r = (A_r, B_r) \oplus C_0 = (A_0, B_0)$
    estará contido em $C_r$:
    \begin{equation*}
        A_r\oplus A_0 \rightarrow \forall a\in A_r \forall a_0\in
        A_0, a+a_0\in A_r
    \end{equation*}

    A afirmação acima é verdade pois, para qualquer
    $r\in\mathbb{Q}$, $A_r = {q\in\mathbb{Q} : q < r}$, e 
    $A_0 = {q\in\mathbb{Q} : q < 0}$. Portanto, pegamos
    um número qualquer menor que $r$ e adicionamos a um
    número negativo qualquer, ou seja:
    \begin{equation*}
        \begin{aligned}
            \forall a\in A_r, \forall a_0\in A_0 \\
            a + a_0 < a \\
            a < r \rightarrow a + a_0 < r \\
            \therefore a + a_0\in A_r
        \end{aligned}
    \end{equation*}

    Da mesma forma temos que como $B_r = {q\in\mathbb{Q} :
    q\geq r}$ e $B_0 {q\in\mathbb{Q} : q\geq 0}$ estermos pegando
    um racional qualquer que seja maior ou igual a $r$ e um 
    racional positivo qualquer ou igual a 0, teremos:
    \begin{equation*}
        \begin{aligned}
            \forall b\in B_r, \forall b_0\in B_0 \\
            b + b_0 \geq b \\
            b \geq r \rightarrow b + b_0 \geq r \\
            \therefore b + b_0 \in B_r
        \end{aligned} 
    \end{equation*}

    \subsubsection*{Elemento Oposto}
    \subsubsection*{Lema 1:}
    Se $C = (A,B)\in\mathcal{F}$, então $\exists -C = (-B,-A)
    \in\mathcal{F}$.

    \subsubsection*{Lema 2:}
    $\forall C\in\mathcal{F}$ temos que $inf(B-A) = 0$

    \subsubsection*{Demonstração}
    Seja $C\in\mathcal{F}$, então $C\oplus-C = C_0$, para provar
    isso, teremos que considerar dois casos distintos:

    \subsubsection*{Caso 1: $inf(B)\notin B$}
    \begin{equation*}
        \begin{aligned}
            \forall a\in A, \forall b\in B \rightarrow a < b \\
            \therefore a - b < 0 \rightarrow a - b\in A_0 
        \end{aligned}
    \end{equation*}
    Para mostrar que $B-A = B_0$ precisamos de um passo a mais,
    $\exists r\in A : -r > 0$, também pelo lema 2 teremos que:
    \begin{equation*}
        \begin{aligned}
            \exists a\in A, b\in B : 0 < b - a < - r \\
            \therefore\exists q\in\mathbb{Q} : -r = q + b - a
        \end{aligned}
    \end{equation*} 

    \subsubsection*{Caso 2: $inf(B)\in B$}
\end{document}