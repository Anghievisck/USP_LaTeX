\documentclass{article}

\usepackage[a4paper, top=2cm, bottom=2cm, left=3cm, right=3cm,
    marginparwidth = 1.75cm]{geometry}

\usepackage[portuguese]{babel}
\usepackage[utf8]{inputenc}
\usepackage{amsfonts}
\usepackage{amsmath}
\usepackage{amssymb}
\usepackage{systeme}

\usepackage{indentfirst}

\usepackage{graphicx}
\usepackage{tikz}

\title{Aulão de IPC}
\author{Anghie}
\date{Junho - 2024}

\begin{document}
    \maketitle

    \tableofcontents
    \newpage

    \section{Funções}
    \subsection*{Definição}
    Uma função em programação não é tão diferente de uma função
    na matemática, ou seja, uma função na programação recebe um
    valor de um tipo, e devolve um valor em algum tipo.
    Mas o que uma função faz em si? Matematicamente falando,
    seja $f$ uma função que pega um valor em $\mathbb{N}$ e
    retorna um valor em $\mathbb{Z}$, ou seja:
    \begin{equation*}
        f(x) = x * (-1)^x
    \end{equation*}

    Em resumo, a função acima pega um $x\in\mathbb{N}$ qualquer
    que se for par vai ser igual a ele mesmo, e se for ímpar
    vai ser igual ao oposto dele.

    Mas o que isso tem haver com programação? Vamos criar a mesma
    função, mas em C:

    \subsection*{Exemplo}
    \begin{figure}[h]
        \centering
        \includegraphics[width=1\textwidth]{paridadeEmC.png}
        \caption{Nossa função em C}
        \label{fig:paridadeEmC}
    \end{figure}
    \newpage

    Como vocês puderam ver, nosso código é um código simples,
    pois ele vai pedir ao usuário um inteiro qualquer,
    e vai falar que o valor desse inteiro será o resultado da
    nossa função. E o que nossa função faz?

    Bom, note que nossa função é do tipo inteiro, ou seja,
    no final da execução dela, no return terá um valor do tipo
    \emph{int}, e o que ela recebe? Dentro do paranteses da
    nossa função está escrito \emph{int x}, o que significa que
    ela recebe um valor qualquer chamado de \emph{x} que é do
    tipo inteiro.
    E como ela funciona? Bom, ela verá se o nosso \emph{x} em
    questão é par, e retornará o próprio \emph{x} se for verdade
    ou o oposto dele, se \emph{x} for ímpar.
    \newpage

    \section{Vetores}
    \subsection{Definição}
    Todos nós sabemos o que é um vetor graças a GA, mas e na
    nossa queirda programação? Bom, uma variável que é um vetor
    é nada mais nada menos que uma variável que pode guardar
    diversos valores, portanto, um vetor em C é basicamente um
    conjunto. Primeiramente, nos lembremos do que é um conjunto:
    
    Um conjunto na matemática é, de maneria simplória, um 
    conjunto é uma junção de números, por exemplo, os 
    $\mathbb{N}$ são todos os números maiores ou iguais a $0$
    sem uma parte racional. Mas a principal e mais importante
    diferença entre um vetor e um conjunto na programação é que
    um vetor pode ser de diferentes tipos, como \emph{char, int,
    float} etc.
    
    Na prática um vetor é uma forma mais fácil de se organizar
    múltiplas variáveis do mesmo tipo, para criar um vetor, 
    primeiro falamos para o computador o nome dele, depois 
    colocamos entre [  ] o tamanho que desejamos para o nosso
    vetor, para facilitar o entendimento, vamos a um exemplo:
    \subsection*{Exemplo}
    \begin{figure}[h]
        \centering
        \includegraphics[width=1\textwidth]{vetoresEmC.png}
        \caption{Vetor em C}
        \label{fig:vetoresEmC}
    \end{figure}

    Como é possivel observar, temos 2 diferentes vetores em
    nosso exemplo, e eles foram criados para receber os valores:
    \emph{"Pedro", 2, 4, 22 e 13}. Note que o número de elementos
    é menor do que o número que coloquei nos vetores, temos
    apenas $5$ elementos, mas temos $9$ elementos no total nos
    vetores. Isso acontece porque no C, uma \emph{string} é
    tratada como um vetor de \emph{char}. Ainda não vimos isso,
    mas uma \emph{string} é uma linha de texto com mais de um
    caractere. Ou seja, o valor "Pedro" é na verdade um vetor de
    $5$ caracteres: ['P','e','d','r','o'].

    Vale a pena saber o conceito de uma \emph{string}, apesar
    dela não existir de fato em C por conta do \emph{\%s}, isso
    diz para o nosso querido \emph{scanf} que ele vai receber um
    valor do tipo \emph{string}, o que permite o não uso do 
    \emph{for}, como é visto para preencher nosso vetor de
    números favoritos, e ele também funciona da mesma forma para
    o \emph{printf}.

    Vale ressaltar que para usar um vetor em C, temos que nos
    lembrar que ele guarda item a item usando um digito como
    identificador. Ou seja, em nosso vetor de números favoritos
    ele terá como identificadores todos os naturais em $[0, 4[$
    ou seja, $0, 1, 2 e 3$, nesta mesma ordem. Isso está sendo
    usado nos \emph{for} acima, onde o valor de \emph{i} também
    é todos os naturais no nosso intervalo de $[0,4[$, em ordem
    crescente. Isso também explica como usamos um vetor. Visto
    que normalmente vamos manipular um elemento por vez nos 
    vetores, para especificar qual, digitamos o nome do vetor
    seguido de [  ] com o indice desejado dentro.
    \newpage

    \section{Matriz}
    \subsection{Definição}
    Uma matriz é praticamente a mesma coisa que uma matriz
    matemática, porém segue as mesmas regras que nossos vetores.
    Ou seja, para criar uma, e preciso nomea-lá e colocar os
    [  ] com o tamanho da matriz, porém ela recebe dois [  ],
    visto que uma matriz tem no mínimo duas dimensões.

    \subsection*{Exemplo}
    Vamos fazer um jogo da velha simples, de uma jogada só, onde
    o \emph{'X'} deve ganhar, para isso, montaremos uma matriz
    quadrada de grau $3$, ou seja, uma matriz $3$x$3$:
    \begin{figure}[h]
        \centering
        \includegraphics[width=1\textwidth]{matrixEmC.png}
        \caption{Matriz em C}
        \label{fig:matrizEmC}
    \end{figure}

    Note que para poder imprimir uma matriz, precisamos de dois
    \emph{for}, visto que igual na matemática, os indices dos 
    elementos de uma matriz são dados pelo formato $ij$. Também
    observe que estamos agora pedindo para o usuário uma
    coordenada $(x,y)$ para que possamos mudar algum elemento
    em especifico da matriz.
    
    Também é preciso notar duas coisas, para acessar um item
    dentro de uma matriz, fazemos um processo parecido com o do
    vetor, dizemos o nome da matriz acompanhado de dois [  ]
    com os indices $ji$, isso mesmo, de forma invertida.
    \newpage

    \section{Struct}
    \subsection{Definição}
    \subsection*{Exemplo}
    \newpage

    \section{Alocação Dinâmica}
    \subsection{Definição}
    \subsection*{Exemplo}
\end{document}