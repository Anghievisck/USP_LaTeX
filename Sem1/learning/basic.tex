% This is essential for every TeX file, this tells LaTeX what is the type
% of document you will create, normally, we use article
\documentclass{article}

% We can choose many formats of paper, and defines its margins and paddings
\usepackage[a4paper, top=2cm, bottom=2cm, left=3cm, right=3cm,
    marginparwidth = 1.75cm]{geometry}

% Here, we have some utils about using especial characters, like in html
% we are using the utf8 pattern, we also are using two special fonts (ams)
% and the language of our file is in english, so, whatever LaTeX autofills
% will end up in english
\usepackage[english]{babel}
\usepackage[utf8]{inputenc}
\usepackage{amsfonts}
\usepackage{amsmath}

% Here it says that every first paragraph will be indented
\usepackage{indentfirst}

% This is used for images, the first one (graphicx) is used to import and
% display images, and tikz is a tool that allows us to draw inside our file
\usepackage{graphicx}
\usepackage{tikz}

% Self-Explanatory
\title{Creating a basic document in \LaTeX{}}
\author{Anghie}
\date{May 2024}

% This defines which folder we are going to use to get our images
\graphicspath{{images/}}

% Basically, all your TeX code will end up here, where you begin creating 
% your file
\begin{document}
    \maketitle

    \tableofcontents
    \newpage
    
    \section{Objectives}
    As I'm learning LaTex, I'll fill this file up with various curious, fun
    and important things, also commenting the code responsable for this file 

    % Here I'm using the graphicx to import the Euler's graphic and also using
    % sections, this helps to organize our pdf and allows a lot of nice features
    % like creating a index in the beginning of our file
    \section{Mathematics}
    \subsection{Euler's Formula}
    This is an attempt to creating a simple paragraph and importing a graphical
    representation as you can see in figure \ref{fig:eulers}, in page
    \pageref{fig:eulers}

    % This is one way to display a image, the "[h]" means "here", so it's shows
    % where it is written in the code, there is also labels, that can be used 
    % as easy ways to reference something
    \begin{figure}[h]
        \centering
        \includegraphics[width=0.5\textwidth]{eulersFormula.png}
        \caption{Euler's Formula}
        \label{fig:eulers}
    \end{figure}

    % A subsection is exactly what it sounds like
    \subsubsection{The most beautiful formula}
    In maths, we do have something called the most beautiful formula, and that
    is Euler's identity minus one:
    % The equation is one of the best characteristics of LaTeX, it allows us
    % to easily write formulas and special characters
    \begin{equation}
        e^{i\pi} -  1 = 0
    \end{equation}
    The actual Euler's Identity is $e^{i\theta} = cos\theta + i sin\theta$, and
    when $\theta = \pi$, we have that:
    \begin{equation}
        e^{i\pi} = cos\pi + isin\pi \rightarrow e^{i\pi} = 1 + 0 \rightarrow
        e^{i\pi} = 1
    \end{equation}
    Ergo the \emph{Euler's Identity}

    \subsection{Limit}
    % \mathbb{} is a function that allows us to use the correct font for something
    % like the ℝ from the Real Numbers
    Given $f$ a function where $f:\mathbb{R} \rightarrow \mathbb{R}$ and $\exists a \in\mathbb{R} \rightarrow f(a) = L$, the $\lim_{x\to a} f(x)$ is given as: 
    \begin{equation}
        \forall\epsilon > 0 \;\: \exists\delta > 0 : \forall x \in\mathbb{R} \rightarrow
        0 < |x - a| < \delta \rightarrow |f(x) - L| < \epsilon
    \end{equation}

    \newpage

    % TikZ, the drawing package, works primarily with coordinates (x,y)
    % it also uses vary keywords 
    \section{Drawing in \LaTeX{}}
    \subsection{Lines}
    \begin{tikzpicture}
        \draw (0,0) -- (4,0);
    \end{tikzpicture}


    \subsection{Parabolas}
    \begin{tikzpicture}
        \draw (0,0) parabola (4,4);
    \end{tikzpicture}


    \subsection{Controling lines}
    \begin{tikzpicture}
        \draw (0,0) .. controls (0,4) and (4,0) .. (4,4);
    \end{tikzpicture}
    % The controls points normaly will make it curvy, acting like a magnet
    % more info at: https://www.overleaf.com/learn/latex/LaTeX_Graphics_using_TikZ%3A_A_Tutorial_for_Beginners_(Part_1)—Basic_Drawing

    \subsection{Rectangles}
    \begin{tikzpicture}
        \draw (0,0) rectangle (4,2);
    \end{tikzpicture}
    % "Wrong" way of drawing a rectangle 
    % \draw (0,0) -- (4,0)  -- (4,4) -- (0,4) -- (0,0);
    % This is the "wrong" way, because we have some tools designed to 
    % make our life easier, e.g. cyle, rectangle, parabola et cetera.

    % For an example, using "cycle", we tell TeX to go back to the
    % original coordinates
    % \draw (0,0) -- (4,0)  -- (4,4) -- (0,4) -- cycle;
    
    \subsection{Circles}
    \begin{tikzpicture}
        \draw (2,2) circle (2cm);
    \end{tikzpicture}
    \newpage

    \section{The Fundamentals of Mathematics}
    \subsection{P1 - USP São Carlos}
    The following subsections will be each question with it's correct answers
    \subsubsection{Question - 1.a}
    For any affirmations $A$ and $B$ the following is an tautology
    \begin{equation}
        (A\rightarrow B) \iff (\neg B \rightarrow\neg A)
    \end{equation}

    Truth Table:
    \begin{equation}
        \begin{tabular}{ c|c|c|c|c|c|c }
            $A$ & $B$ & $\neg A$ & $\neg B$ & $A \rightarrow B$ & $\neg B \rightarrow \neg A$ & $(A \rightarrow B) \iff (\neg B \rightarrow \neg A)$ \\
            \hline
            $T$ & $T$ & $F$ & $F$ & $T$ & $T$ & $T$ \\
            $T$ & $F$ & $F$ & $T$ & $F$ & $F$ & $T$ \\
            $F$ & $T$ & $T$ & $F$ & $T$ & $T$ & $T$ \\
            $F$ & $F$ & $T$ & $T$ & $T$ & $T$ & $T$ \\
        \end{tabular}
    \end{equation}

    \subsubsection{Question - 1.b}
    For any affirmations $A$, $B$ and for any absurd $C$ the following are tautologies 
    \begin{equation}
        (\neg A\rightarrow C) \rightarrow A \text{ and } [(A \wedge \neg B) \rightarrow C] \rightarrow (A \rightarrow B)
    \end{equation}

    First truth table:
    \begin{equation}
        \begin{tabular}{ c|c|c|c|c }
            $A$ & $C$ & $\neg A$ & $\neg A \rightarrow C$ & $(\neg A \rightarrow C) \rightarrow A$ \\
            \hline
            $T$ & $F$ & $F$ & $T$ & $T$ \\
            $T$ & $F$ & $F$ & $T$ & $T$ \\
            $F$ & $F$ & $T$ & $F$ & $T$ \\
            $F$ & $F$ & $T$ & $F$ & $T$ \\
        \end{tabular}
    \end{equation}

    Second truth table:
    \begin{equation}
        \begin{tabular}{ c|c|c|c|c|c|c|c }
            $A$ & $B$ & $\neg B$ & $C$ & $A \wedge\neg B$ & $(A \wedge\neg B) \rightarrow C$ & $A \rightarrow B$ & $[(A \wedge\neg B) \rightarrow C] \rightarrow (A \rightarrow B)$\\
            \hline
            $T$ & $T$ & $F$ & $F$ & $F$ & $T$ & $T$ & $T$ \\
            $T$ & $F$ & $T$ & $F$ & $T$ & $F$ & $F$ & $T$ \\
            $F$ & $T$ & $F$ & $F$ & $T$ & $F$ & $T$ & $T$ \\
            $F$ & $F$ & $T$ & $F$ & $F$ & $T$ & $T$ & $T$ \\
        \end{tabular}
    \end{equation}
\end{document}