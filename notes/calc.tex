\documentclass{article}

\usepackage[a4paper, top=2cm, bottom=2cm, left=3cm, right=3cm,
    marginparwidth = 1.75cm]{geometry}

\usepackage[portuguese]{babel}
\usepackage[utf8]{inputenc}
\usepackage{amsfonts}
\usepackage{amsmath}
\usepackage{amssymb}
\usepackage{systeme}

\usepackage{indentfirst}

\usepackage{graphicx}
\usepackage{tikz}

\title{Anotações de Cálculo}
\author{Anghie}
\date{May 2024}

\begin{document}
    \maketitle

    \tableofcontents
    \newpage

    \section{Limite}
    \subsection{Definição}
    O limite é uma forma de conseguir encontrar um valor
    proximo a um outro valor onde uma função dê um resultado
    proximo. Ou seja, definimos uma distância para o resultado
    ($\epsilon$) que nos fala uma distância para a entrada
    ($\delta$).
    \begin{equation*}
        f(a) = L \rightarrow \lim_{x\to a} f(x) = L
    \end{equation*}

    Assim, matemáticamente falando, podemos falar que o limite é
    definido como:
    \begin{equation*}
        \forall \epsilon > 0 \; \exists \delta > 0 : \forall x
        \in \mathbb{R} \rightarrow 0 < |x - a| < \delta \rightarrow
        |f(x) - L| < \epsilon
    \end{equation*}
    \subsection{Resolução}
    Para resolver um limite, tentamos substituir $x$ por $x_0$
    porém o valor de $x$ nunca seria de fato igual a $x_0$, o que
    permite por exemplo:
    \begin{equation*}
        \lim_{x\to 0} \frac{2x + 1}{x} = \infty
    \end{equation*}
    no caso acima claramente vemos que ao substituir $x$ por
    $x_0 = 0$ iriamos ter que $\frac{2(0) + 1}{0}$, o que na
    matemática normal é indefinido, porém como o $x$ só se aproxima
    de $0$, na realidade não estamos dividindo por $0$, e sim por um
    número muito próximo a $0$, o que permite a realização da conta

    \newpage
    \section{Derivada}
    \subsection{Definição}

    
    \subsection{Resolução}
    A derivada é calculada da seguinte forma:
    \begin{equation*}
        \lim_{x\to x_0} \frac{f(x) - f(x_0)}{x - x_0}
    \end{equation*}

    Também podemos falar que $h = x - x_0$, portanto, podemos
    mostrar a derivada como:
    \begin{equation*}
        \lim_{h\to 0} \frac{f(x+h) - f(x)}{h}        
    \end{equation*}
    $h\to 0$ pois $x\to x_0$ e $h = x - x_0$

    \subsection{Regra da Cadeia}
    Toda derivada pode ser representada da seguinte forma:
    \begin{equation*}
        \frac{df(g(x))}{dg(x)} \cdot \frac{dg(x)}{dx}
    \end{equation*}

    No caso de $g(x)$ ser igual a $x$, então a derivada é igual a:
    \begin{equation*}
        \frac{df(x)}{dx} \cdot \frac{dx}{dx} = \frac{df(x)}{dx}
        \cdot 1 = \frac{df(x)}{dx}
    \end{equation*}

    \subsection{Regra do Tombo}
    A regra do tombo é uma das mais usadas, junto da regra da
    cadeia, ela pega, por exemplo: $f(x) = ax^n$ e através de um
    "tombo", conseguimos a derivada de $f(x)$, como podemos ver
    abaixo:
    \begin{equation*}
        \lim_{x\to x_0} \frac{df(x)}{dx} = nax^{n-1}
    \end{equation*}
    caso, queiramos derivar novamente, o tombo permite o seguinte:
    \begin{equation*}
        \frac{d^2f(x)}{dx^2} = (n-1) \cdot nax^{(n-1)-1} = (n-1)
        \cdot nax^{n-2}
    \end{equation*}
    Aqui podemos identificar um padrão, $\forall k \in \mathbb{R}$
    sendo $k \le n$:
    \begin{equation*}
        \frac{d^kf(x)}{dx^k} = [\prod_{j = 0}^{k - 1} (n - j)]
        \cdot x^{(n - k)}
    \end{equation*}

    \subsection{L'Hôpital}
    O teorema de L'Hôpital é um dos teoremas mais amados pelos
    estudantes, visto que ele permite resolver limites
    indeterminaveis usando derivadas, ou seja, quando no limite
    obtemos $\frac{0}{0}$ ou $\frac{\infty}{\infty}$
    \begin{equation*}
        \lim_{x\to 1} \frac{x^2 - 1}{x^2 - 2x + 1} \stackrel{LH}{=}
        \lim_{x\to 1} \frac{2}{2x - 2} = \frac{2}{0} = \infty
    \end{equation*}

    Se fossemos resolver o mesmo limite sem o uso de L'Hopital
    teriamos que fazer:
    \begin{equation*}
        \lim_{x\to 1} \frac{x^2 - 1}{x^2 - 2x + 1} =
        \frac{(x + 1)(x - 1)}{(x - 1)^2} = \frac{x + 1}{x - 1}
        = \frac{2}{0} = \infty
    \end{equation*}

    \subsection{Regra do Produto}
    Ao derivarmos um produto devemos realizar a seguinte fórmula:
    \begin{equation*}
        \frac{df(x)\cdot g(x)}{dx} = \frac{df(x)}{dx}\cdot g(x) +
        f(x) \cdot\frac{dg(x)}{dx}
    \end{equation*}

    \subsection{Regra do Quociente}
    Se quisermos derivar uma fração qualquer, como por exemplo a
    $\frac{f(x)}{g(x)}$ teremos como sua derivada a seguinte
    equação:
    \begin{equation*}
        \frac{d\frac{f(x)}{g(x)}}{dx} = \frac{\frac{df(x)}{dx}
        \cdot g(x) - f(x)\cdot\frac{dg(x)}{dx}}{g^2(x)}
    \end{equation*}

    Exemplo: $f(x) = \frac{1}{x}$, considere $1 = f(x)$ e $x = g(x)$
    substituindo na equação acima temos:
    \begin{equation*}
        \frac{d}{dx}(\frac{1}{x}) = \frac{\frac{d}{dx}1 \cdot x - 1
        \cdot \frac{d}{dx}x}{x^2} = \frac{0 \cdot x - 1 \cdot
        1}{x^2} = -\frac{1}{x^2} 
    \end{equation*}

    Porém, neste caso poderiamos ter usado a regra do tombo, pois
    $f(x) = \frac{1}{x} = x^{-1}$, logo:
    \begin{equation*}
        \frac{df(x)}{dx} = \frac{dx^{-1}}{dx} \rightarrow - 1 \cdot
        x^{-1 - 1} = - x^{-2} = -\frac{1}{x^2}
    \end{equation*}

    \subsection{Máximos e Mínimos Locais}
    Um máximo/mínimo local será um ponto $c$ em um intervalo $]a,b[$
    dentro do domínio da função onde $\nexists x \in\; ]a,b[ \;\in
    dom(f)$ tal que $\stackrel{\text{máx}}{f(c)}\; > f(x)$ ou
    $\stackrel{\text{mín}}{f(c)}\; < f(x)$. Conseguimos descobrir os
    máximos e mínimos locais através dos pontos críticos da função
    que a derivada dá.

    \subsubsection*{Pontos Críticos}
    Para descobrir os pontos críticos de uma função, primeiro
    derivamos a mesma, depois tentamos obter as raízes da derivada

    \subsection{Concavidade}

    \subsection{Assíntotas}

    \section{Integral}
    A integral de uma função é uma forma de descobrir qual a área da
    função em certo intervalo. Seja $f(x)$ dada pelo seguinte gráfico
    e o intervalo seja $[a,b]$, uma representação simplória da
    integral seria:

    \begin{equation*}
        \begin{tikzpicture}
            \draw (-2, 0) -- (2, 0); 
            \draw (0, -1) -- (0, 2); 

            \draw[red, thick](0, 0.5) -- (0, 1);
            \draw[red, thick](0, 1) -- (1, 1);

            \draw[blue, dashed](1, 0) -- (1, 1);
            \draw[blue, thick](-2, -0.5) -- (-1, 0) -- (0, 0.5) --
            (1, 1) -- (2, 1.5);

            \draw[green, dashed](0, 0) -- (0, 0.5);
            \draw[green, dashed] (0.5, 0) -- (0.5, 0.75);
            \draw[green, thick] (0, 0.75) -- (0.5, 0.75);

            \draw[red, fill, opacity = 0.2](0, 0.5) -- (0, 1) --
            (1, 1) -- (0, 0.5);
            \draw[red, fill, opacity = 0.2] (0.5, 0.74) -- (0, 0.74)
            -- (0, 0.5);

            \draw[blue, fill, opacity = 0.2](0.5, 0) -- (1, 0) --
            (1, 1) -- (0.5, 0.75) -- (0.5, 0);

            \draw[green, fill, opacity = 0.2](0, 0) -- (0.5, 0) --
            (0.5, 0.75) -- (0, 0.5) -- (0, 0);

            \node[blue] at (-0.15, -0.15) {\small $a$};
            \node[blue] at (1, -0.15) {\small $b$};
            \node[blue] at (0.9, 1.2) {\footnotesize $f(b)$};

            \node[green] at (0.5, -0.15) {\small $x_0$};
            \node[green] at (-0.4, 0.8) {\footnotesize $f(x_0)$};
        \end{tikzpicture}
    \end{equation*}

    A formula da área, $A$, no intervalo $[a, b]$ de $f(x)$ é:
    \begin{equation*}
        (b - a) \cdot\int_{a}^{b} f(x)dx = A 
    \end{equation*}

    \subsection{Teorema Fundamental do Cálculo}
    Seja $\mathbb{I}$ intervalo fechado, limitado e que seja infito.
    Sejam $f: \mathbb{I}\rightarrow\mathbb{R}$ contínua e $F:
    \mathbb{I}\rightarrow\mathbb{R}$ uma função. São equivalentes.
    \begin{equation*}
        \text{a) } \exists a\in\mathbb{I}:F(x) = F(a) +
        \int_{a}^{x}f(t)dt
    \end{equation*}
    \begin{equation*}
        \text{b) } F \text{ é derivável e } F'(x) = f(x) \forall x
        \in\mathbb{I} \text{ onde $x$ não está nas extremidades de }
        \mathbb{I}
    \end{equation*}

    \subsubsection*{Demonstração}
    Vamos provar que o item b implica o item a. Considere $G$, $f$ e
    $F$
    \begin{equation*}
        G(x) = \int_{a}^{x}f(t)dt
    \end{equation*}
    Pela primeira parte:
    \begin{equation*}
        G'(x) = f(x)
    \end{equation*}
    \begin{equation*}
        [F(x) - G(x)]' = f(x) - f(x) = 0
    \end{equation*}
    Logo $G$ e $F$ se diferem por uma constante. Note que $G(a) = 0$.
    Ou seja a constate é $F(a)$ 
    \begin{equation*}
        \therefore F(x) = F(a) + G(x) = F(a) + \int_{a}^{x}f(t)dt
    \end{equation*}

    hahahahahahaha
\end{document}